\documentclass[final]{polytech/polytech}
% zone du préambule
\schooldepartment{di}
\typereport{custom}
\typereportname{Projet ASR}
\reportyear{2017-2018}

\title{Configuration d’un OS pour le calcul parallèle}

%\reportlogo{images/logolinkr}

\student[di5]{Benjamin}{Caldas}{benjamin.caldas@etu.univ-tours.fr}
\student[di5]{Logan}{Verecque}{benjamin.caldas@etu.univ-tours.fr}

\academicsupervisor[di]{Patrick}{Martineau}{patrick.martineau@univ-tours.fr}

\motcle{Système d'exploitation}
\resume{Projet ASR}

\keyword{Operating system}
\abstract{ASR project}

\addbibresource{biblio}

\usepackage{ulem}

\begin{document}

\part{Introduction}

%\vspace{1\baselineskip} %saut de ligne

%\vfill\eject %saut de page

%\begin{figure}
%	\pgfimage[width=5cm]{images/.jpg}
%	\caption{}
%	\label{fig:}
%\end{figure}

\section{Contexte}
5A projet ASR (Architecture Système et Réseaux)

\section{Sujet}
La figure suivante contient l'énoncé exact de notre sujet.

\begin{figure}
	\pgfimage[width=12cm]{images/sujet_asr.png}
	\caption{Énoncé exact du sujet}
	\label{fig:enonce_sujet_projet_asr}
\end{figure}

\section{Objectifs}
L’objectif de notre projet était de mettre en place un système d’exploitation Unix contenant tous les outils permettant de réaliser du développement suivant les trois grands axes de parallélisations :

- OpenMP

- Nvidia GPU Cuda

- MPI

Ce système devait également être portable et bootable depuis n’importe quel clé usb. Cette contrainte indique également que le système doit être léger, simple d’utilisation et d’installation.

\part{Définition du sujet}

\section{Système d'exploitation}
Un système d'exploitation (SE), ou Operating System (OS) en anglais est ...

\section{Clé USB bootable}
Une clé USB bootable est une clé USB qui intègre un OS permettant de démarrer un ordinateur sans faire appel à un autre élément de stockage. 

Parler du Grub

\section{OpenMP}
Permet le calcul parallèle

Fonctionne avec le compilateur gcc

S'appuie sur le CPU

\section{Cuda}
Permet le calcul parallèle

S'appuie sur GPU de marque NVIDIA

Fonctionne avec le compilateur nvcc

\section{MPI}
Permet le calcul parallèle

S'appuie sur la communication en réseau

\part{Solution envisagée}

\section{Principe}
S'appuyer sur un OS léger car clé de 16 Go max.

Lui installer les éléments nécessaires

Permettre d'utiliser les 3 outils sans aucune action nécessaire si le contexte le permet.

OpenMp -> toujours

Cuda -> si carte graphique NVIDIA

MPI -> si réseau

\section{Choix du système d'exploitation}

\subsection{Recherche}
Le choix de l’OS était totalement libre, l’idée était néanmoins d’avoir un système assez léger.

\subsection{Lubuntu}
Léger, simple et réactif.

\part{Étapes de réalisation}

\section{Création de la clé bootable}

\subsection{Création de l'environnement}
Environnement live Lubuntu sur clé de faible capacité (4 Go suffisent).

\subsection{Installation et partitionnage de la clé USB}
Environ 1 Go pour Lubuntu

4 Go swap, 

reste en ext2 pour le stockage (la persistance)

\subsection{Installation}

\subsection{Validation OpenMP}

\section{Ajout de Cuda}

\subsection{Vérifications}

\subsection{Installation Cuda}

\subsection{Installation nvcc}

\subsection{Validation}

\section{MPI}

\subsection{Installation}

\subsection{Validation}

\part{Gestion de projet}

\section{Organisation}
2 créneaux par semaine, découpage entre nous deux

\section{Planning}
Faire un Gantt très arrangé

\section{Gestion de versions}
Les .img

\part{Problèmes rencontrés}

\section{Connexion réseau de Polytech}
Impossible de télécharger les paquet sur 2h tellement le débit est faible.

\section{Temps des installations}
limite pour des créneaux de 2h

gcc + cuda + nvcc = plus de 6h avec une bonne connexion

\section{Choix de du système d'exploitation}

\section{Secure Boot}
Mode UEFI, Legacy

Fucking Asus -> secure boot mode abusé

\section{Mode Live et mode persistant}

\section{Persistance limitée}
Les 4 Go max avec Linux Live Creator

\section{Limite de stockage}
Une clé de 16 Go c'est peu

plus beaucoup d'espace pour les développeurs

\section{Fragilité}
Veillez à ce que la clé soit bien connectée sinon au revoir l'OS, d'où l'utilité des sauvegardes qu el'on aurait dû mettre en place bien avant...

\part{Outils utilisés}
Dans cette partie, nous récapitulerons tous les outils utilisés, nous détaillerons leurs fonctionnalités et nous indiquerons où et comment se les procurer.

\section{Linux Live Creator}
Création clé live de départ

\section{USB Image Tools}
Sauvegarde des .img

\section{Etcher}
Ré-injection des .img

\section{Diskpart}
Re-partitionnement de la clé USB.

\section{Github}
Pour la rédaction du rapport

\part{Guide du développeur}
Dans le but d'aider un futur développeur qui se lancerait dans la même tâche que celle qui nous a été confiée, nous avons créé un guide destiné à une personne ayant des compétences assez avancées en système dans le but de lui donner la marche à suivre.

Ici ou en annexe

\part{Conclusion}

\end{document}